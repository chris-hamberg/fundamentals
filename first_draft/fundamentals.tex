\documentclass{article}
\usepackage[margin=0.5in]{geometry}
\usepackage{amsthm}
\usepackage{xcolor}
\usepackage{alltt}
\usepackage{relsize}
\usepackage{graphicx}
\graphicspath{{}}
\raggedbottom
\raggedright
\pagecolor[rgb]{0.0, 0.0, 0.0}
\color[rgb]{1, 1, 1}
\title{All Shill Narratives Destroyed by Easy Math}
\author{Chris Hamberg}
\date{}

\begin{document}
\maketitle
\thispagestyle{empty}
There is a great deal of confusion in the AMC shareholder community. The shill 
narratives are: The Debt Narrative, The Dilution Narrative, The Save Billy 
Narrative, and the Short Thesis. Why? Because they consist of fear, uncertainty, 
and doubt. Now, before going into the DD, I'd like to point out some 
loosely related statements that are unrelated to the strength of this DD, but are 
none-the-less very interesting. We trust
our CEO, so they must be true statements when Mr. Aron said (in Phoenix):
\begin{alltt}
1) "We (AMC) have plenty of money (as it stands right now.)"
2) To paraphrase: the ONLY reason Adam wants you to vote yes for RSC is because he, "can use the 
   dilution to raise capital, therefore paying down debt to improve the fundamentals."
\end{alltt}
Now there are a few other statements he gave (in Phoenix) that reveal key 
information about AMC from which we can infer an accurate estimation on the real 
AMC fundamentals. Mr. Aron has said the following, verbatim:
\begin{alltt}
1) "the (domestic) box office fell, in from 2019, 11.4 billion,... , to 2 billion."
2) "the number of movies that major Hollywood studios released in 2019 it was around 120..."
3) "they (Hollywood) released in 2022... ...72"
4) He said Q1 + Q2 + Q3, alone, the domestic box office generated 7.4 billion
5) He said that 2023 projected a 35% increase in Hollywood releases.
6) He also said that 2023 projected 102 new Hollywood releases (these are not the same number.)
\end{alltt}
Notice that none of these data points account for the distribution of 
revenue between theater brands. Regal is bankrupt. So it's reasonable to expect
that AMC got a larger percentage of domestic box office market share than it has 
in the past, compared to 2022, and will continue to do so in the future because of 
The Logical Principle of The Uniformity of Nature (you might think we are not 
speaking about nature when talking about numbers following from Regal bankruptcy, 
but we are because computation is natural!) This is important, 
because we will use this in conjunction with The Transitivity Law from the Order 
Axioms for Real Numbers to construct valid inferences, shortly (not to worry, the 
math is just very simple arithmetic.)
\newline \newline
Also, notice that 2019 revenue for AMC was approximately \$1.448B. At these levels 
AMC has an annualized cash burn of just under \$14M, but at 2019 cost of revenue 
levels (which is nearly doubled compared to 2022 cost of revenue.)
\newline \newline
First, take note of a simple mathematical fact so that you can understand how 
the following inferences are valid and correct applied mathematics. From 
statistics there is a theorem called The law of Large Numbers. Simply put, it means 
response variables converge with their average. For example, if you actually flip an 
equally weighted coin an infinite number of times, the outcomes really will tend to 
the center (50\% heads, 50\% tails) by The Law of Large Numbers. This kind of 
distribution is naturally built into the relationship between movie releases and 
revenues. That means we can reliably predict revenue from movie releases, provided 
a significant black swan does not disrupt economic conditions. We are talking about 
a black swan of enormous impact that is not likely to occur. Thus, it is extremely 
reasonable to infer revenue projections from release numbers, and such is almost 
certainly a reliable predictor (provided that a significant black swan disruption 
does not occur.) Mr. Aron is aware of this; which is why he has been referencing 
it in his speaches.
\newline \newline
First we connect some corresponding datapoints from what Mr. Aron said:
\begin{alltt}
1) In 2019 there were approximately 120 Hollywood releases, and the domestic box office generated 
   \$11.4 billion.
2) In 2022 there were approximately 72 Hollywood releases, and the domestic box office generated 
   \$7.4 billion (excluding Q4)
\end{alltt}

\textit{(We can estimate Q4 domestic box office and 2022 total domestic box office. 
$\frac{7.4}{3} \approx 2.47. \quad 2.47 \cdot 4 \approx 9.88$)}
\newpage
\thispagestyle{empty}
Now we have two ratios with which we can make a comparison.
Let's look at 2019. $\frac{\$11.4B}{120} \approx \$11.67M$ of revenue in domestic 
ticket sales per movie on average. 2022 excluding Q4:
$\frac{\$7.4B}{72} \approx \$102.78M$ of revenue in domestic ticket sales per 
movie on average! This may appear to violate The law of Large Numbers, however 
consider that Covid was a black swan. Evidently Covid had a positive effect on 
domestic box office revenues. Per Hollywood release domestic 
ticket sale revenues increased by more than an order of magnitude! We also left 
Q4 out. Using the Q4 estimate: $\frac{\$9.88}{72} \approx \$137.22M$ in domestic 
box office per release ticket sales! Incredible! That means if 120 movies had 
been released in 2022, the total domestic box office would have been more than 
\$12.33B using the \$7.4B datapoint; nearly a \$1 billion dollar increase 
compared to 2019 domestic box office! Applying the same logic using the Q4 
estimate that would be \$16.47B in domestic box office!
\newline \newline
So, what percentage of the 2022 domestic box office went to AMC? We construct an 
estimated percentage market share of domestic box office. In 2019, AMC sold 62.32M 
tickets domestically at an average ticket price of \$9.47 for a total of \$590.16M! 
As a percentage of total domestic box office that is a 5\% market share. Which  
accounts for 40.77\% of 2019 total revenues for AMC (because it doesn't include 
revenue from global box office nor from concessions). From here on we will use the 
5\% market share of domestic box office, and the 40.77\% of that domestic box 
office market share as the proportion of AMC total revenues: each as estimators.
\newline \newline
Consider the following hypothetical case where we pretend 120 major Hollywood movies 
had been released in 2022. This will give us a normalized comparison between 2019 
and 2022 AMC real business fundamentals, and real economic conditions, in terms of 
total revenue. Remember when I said Regal is bankrupt. So it's safe to assume that 
in 2022 AMC got at least as much market share of the domestic box office since 
their competitor must have gotten less than their 2019 levels. So AMC got at least 
5\% of the domestic box office in 2022. We use this estimator such that it biases 
against the bullish case, then. 5\% of \$12.33B is \$616.5M in domestic 
box office. Let's extrapolate 2022 estimated total revenue using the total revenue 
estimator constructed in the above paragraph. If \$616.5M is 40.77\% of revenue, 
then total revenue must be
$\frac{\$616.5M}{40.77\%} = \frac{\textit{total revenue}}{100\%}$, that is 
over \$1.5B, if 120 movies had been released in 2022. So at these levels 
AMC would easily be profitable and cash flow positive. How can we know? In 2019 
AMC had \$1.448B in revenue with a net loss of \$13.5M. In this hypothetical 
scenario, AMC would have been net profitable by approximately \$50M at 2019 
expense levels! But cost of revenue has been reduced by nearly half since that 
time! So it must be more than \$50M! Now consider the case in which we include 
the Q4 estimate to estimate the total 2022 domestic box office: \$9.88B, such 
that the per release average domestic box office revenue would be \$137.22 per 
release. At 120 releases that would mean the domestic box office for 2022 would 
have generated \$16.47B (nearly 50\% growth. Amazing!) At 5\% market share that 
would give AMC \$823.32M in domestic ticket sale revenues, and an estimated total 
revenue of just under \$2.02B dollars! Nearly enough cash flow to service 10\% of 
the debt per year, and again, that is assuming 2019 expense levels. So the profit 
margin and ability to service the debt would really be much greater than that!
This means real AMC fundamentals have never been so strong! 
\newline \newline
So let's see the actual 2023 projection. We look at projected revenue as the
informal derivative of projected new releases. New releases are projected to 
increase by 35\% in 2023 compared to 2022. 35\% of 72 releases is 25.2 additional 
releases for a total of 97.2 releases. We round down to bias against the strength 
of the conclusion. So 97 releases. (Also note that Mr. Aron said 102 releases in 
2023, 
but since he said 35\% and 102, we will first assume 35\% to bias projected 
revenue towards the bearish case.) $97 \cdot \$102.78M \approx \$9.97B$. 
That's, domestic box office. Again, AMC makes 
at least 5\% of the market share in the domestic box office. So projected 
revenue increases in domestic box office alone for AMC is $0.05 \cdot \$9.97B 
\approx \$498.5M$. Again, we extrapolate: 40.77\% of total revenue comes from 
domestic box office. So total extrapolated and projected revenue must be 
approximately \$1.22B in 2023. 2022 up to and including Q3 financials report 
\$968.4M in 
revenue with net loss of \$266.9M. Remember that net loss number. 
$\$1.22B - 968.4M = \$253M$. 
\$266.9M - 253M = \$13.9M annualized deficit projected in 2023; a \$3.48M 
quarterly short fall. Mr. Aron wants to dilute shareholders up to 5B shares to 
make up for that short fall. Remember We never included Q4 domestic box office. 
Let's include the 2022 Q4 estimate to calculate a more likely 2023 projection. 
$97 \cdot \$137.22M \approx \$13.31$ domestic box office for 2023. With a 5\% 
market share: \$665.5M in domestic ticket sale revenues for AMC. Using the total 
revenue estimator the projected 2023 
total revenue for AMC would be \$1.632B. Putting AMC well above profitable levels
even with 2019 levels cost of revenue (which is now nearly half of what it was 
in 2019). Now finally, let's use both 102 projected releases for 2023 (which I 
expect to be 
a more truthful claim made by Mr. Aron since it is an exact number, and not an 
abstraction unlike his 35\% claim,) and the 
upper end per release average revenue (based on including the Q4 estimate) to 
predict the bullish (and very realistically likely) case: 
$102 \cdot \$137.22M \approx \$14B$ domestic box office (over 27\% growth compared 
to 2019.) With 5\% market share that puts AMC domestic ticket sale revenues at
\$700M, using the estimator we get \$1.72B in total projected revenue in the 
bullish case. Absolutely crushing all shill narratives! That is the Debt
narrative, Dilution Narrative, Save Billy Narrative, and the Short Thesis ALL 
annihilated by simple mathematical analysis! AMC to the moon!
\begin{alltt}

    
Data Source: AMC Investor Relations (https://investor.amctheatres.com/financial-performance/)
\end{alltt}


\end{document}